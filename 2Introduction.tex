\chapter{General Introduction}

The story that I want to tell with this thesis is the following:

\begin{itemize}
  \item What is the purpose of modeling? 
    Short answer: exploration of qualitative properties of real world systems and the ability to answer 'what if' questions.
    Long answer: along the lines of this:
    \begin{itemize}
      \item \emph{Models in SES are not what they are in physics.}
    The premises of the scientific method are that there exists an objective reality, that is reality is governed by natural laws and that these laws are immutable. In physics, models are use to formalize and narrate natural laws. Consequently, models tend to be regarded as an exact description of reality. However, if one wants to model ecological or even social-ecological systems, due to their immense complexity, modeling every entity involved according to its natural laws is futile. Consequently, models of SES usually require heavy simplifications of and abstractions from their real world counterparts and therefore, in this case the view of models being good if they are an exact description of the real world is bound to hit the wall\footnote{Or Igor Sokolov used to put it in his lectures: ``the best model for a cat is a cat. Preferably the same cat'' To be imagined in a heavy Russian accent.}.
  \item \emph{SES models are more like models of complex systems.}
    It helps to view SES from a complex systems perspective. In modeling complex systems it has proven desirable to only include the model entities and processes that are needed to give rise to the phenomena of interest. This parsimony as a guiding principle for model development is also known as Occams razor.
  \item \emph{What do complex system models have to do with the real world? or: What is the purpose of SES models in epistemological terms?}
    The abstraction of complex systems models provokes the question: How do these models relate to their real world counterparts and what can one learn about cause and consequence in the real world from studying their equivalents in such a stylized model?
    There are different approaches to this question:
    \begin{itemize}
    \item 1) models are conceptual explorations of different possibilities in a system (Hausmann 1992)
    \item 2) models of the real world are always false. Deal with it. Oreskes1994
    \item 3 )modeling selects particular parts of a system so study their workings in isolation. (M\"{a}ki 1992)
    \item 4) models are 'credible worlds' and inferences are to be treated accordingly: If the world were like that, it would be such and such. Given that the real world is similar to this 'credible world' the inferences may fuzzily apply. (Sudgen2000)
  \item 5) Similarly, models are credible surrogate systems according to (M\"{a}ki2009)
    \end{itemize}
    These accounts help to understand what can be learned from complex systems models, namely, that they facilitate the exploration of dynamical properties and their dependence on external parameters and as such help to generate qualitative insights on how the real-world system works.
\end{itemize}
\item \emph{what does this mean for models of archeological and historical processes?}

\item \emph{how can models produce new insights?} Something about complex systems, agent-based models and emergence.

\item \emph{Agents in models are limited to the behavioral rules that specify them. Given that, how can societal change be modeled?} Propose the separation of perception, decision making and actions in modeling (human) agents. Examples would be reinforcement learning or heuristic decision making.

\item \emph{What about the bounded cognitive capabilities of real humans?} Elaborate on bounded rationality and fast and frugal heuristics as one possible implementation. Also note the interpretation of cue orders as norms in the context of preferential (vs. inferential) decisions.

\end{itemize}


This general introduction has to cover the following topics:
\section{Agent-Based Modeling}
Definitely.
\section{`World-Earth' Modeling}
Also has to be in there. But not sure w.r.t. to the framing and wording.
With reference to the Copan Core framework paper
\section{Heuristic Decision Making}
Definitely. With reference to the section in Finns review paper
\section{Adaptive Networks and Aggregation Methods?}
Not so sure. But I can still decide on that later.
