\selectlanguage{english}

\begin{abstract}
    \begin{minipage}[t][0pt]{\linewidth}
  % Motivation
  The trajectory of the Earth system in the Anthropocene is governed by an increasing entanglement of processes on a physical and ecological as well as on a socio-economic level. At the same time, humanity is facing a number of substantial challenges in sustainably navigating this system such as anthropogenic climate change, rapid degeneration of biosphere integrity and increasing economic inequality. 
  %In the face of this it becomes more and more apparent that in order to avoid catastrophic environmental change while staying within a desirable social regime, rapid changes in society and economy are necessary. To find ways to navigate possible scenarios for these changes, highly sophisticated integrated assessment models are in use. These models make very strong assumptions about human decision making and behavior representing diverse human actors in terms of a representative agent. This approach, while technically convenient, does not allow for the description of emergent phenomena such as self-organization or tipping. However, there is ample historical evidence that large scale social changes in society and economy such as the abolition of slavery or woman's rights were the product of social movements i.e. of self-organized collective action of heterogeneous decision makers in a diverse society.  
  If models are to be useful as decision support tools in this environment, they ought acknowledge these complex feedback loops as well as the inherently emergent and heterogeneous qualities of societal dynamics.
  % Mission statement
  This thesis takes different angles to improve the capability of social-ecological and socio-economic models to picture emergent social phenomena
  and uses and extends techniques from dynamical systems theory and statistical physics for their analysis
.

  %Mayasim Study
  It begins with a modeling study of the social-ecological system of the ancient Maya on the Yucatan peninsula. This study analyzes the possible endogenous dynamics resulting from local population growth that is sustained by income agriculture and ecosystem services, followed by the over-usage of environmental resources, resulting in loss of income and consequently migration, decline in local population and spatial reorganization. The study shows that drought events of severity and duration that are in line with paleoclimatic data are not capable to cause lasting changes in the Maya civilization in the model. This is in line with an existing literature that argues that in addition to climatic stress, internal societal changes must have had occurred to produce the large scale catastrophic decline and reorganization of the Maya population.% during the period of the terminal classic.

  % Social learning of bounded rational decision heuristics
  As one possible way to model endogenous societal changes, this thesis proposes the differentiation of judgement and action in human decision making. Particularly, it proposes to model humans as bounded rational decision makers that use (social) learning to acquire decision heuristics that function well in a given environment.
  % Divestment Study
  Subsequently, this thesis presents a two sector economic model in which one sector uses a fossil resource for economic production while the other uses fossil free technologies. In this model, households make their investment decisions in the previously proposed way. The model's parameters are fitted to historical data and the model dynamics are analyzed in a series of numerical experiments. These experiments show how in the model economy individual decision making and social dynamics can not limit CO$_2$ emissions to a level that prevents global warming above $1.5^{\circ}$C. However, they also show that a combination of collective action and coordinated public policy actually can.

  % Savings
  A follow up study analyzes social learning of individual savings rates in a one sector investment economy. Here, households are embedded in a static social network and set their savings rate by imitating their neighbor with the highest consumption. It shows that households are undersaving if the interaction rate in the social learning process is very high, but that the aggregate savings rate in the economy approaches that of an intertemporarily optimizing omniscient social planner if the interaction rate decreases. Also, a decreasing social interaction rate leads to emergent inequality in the model in the form of a sudden transition from a unimodal to a strongly bimodal distribution of wealth among households.

  % Approximation
  Finally, this thesis proposes a combination of different moment closure techniques that can be used to derive analytic approximations for networked heterogeneous agent models such as the ones used in this thesis where interactions between agents occur on an individual as well as on an aggregated level.
  \end{minipage}
\end{abstract}

\cleardoublepage

\selectlanguage{ngerman}
\begin{abstract}
  Die Entwicklung des Erdsystems im Anthropoz\"{a}n wird durch eine zunehmende Verflechtung von Prozessen sowohl auf physikalischer und biologischer als auch auf sozio\"{o}konomischer Ebene bestimmt. Gleichzeitig steht die Menschheit bei der nachhaltigen Steuerung dieses Systems vor einer Reihe von gro{\ss}en Herausforderungen wie z.B. dem anthropogenen Klimawandel, der fortschreitenden Degeneration der Integrit\"{a}t der Biosph\"{a}re sowie zunehmender wirtschaftlicher Ungleichheit.

  Wenn Modelle als Entscheidungshilfe in diesem Umfeld n\"{u}tzlich sein sollen, sollten sie diese R\"{u}ckkopplungsschleifen sowie die inh\"{a}rent emergenten und heterogenen Qualit\"{a}ten gesellschaftlichen Prozesse ber\"{u}cksichtigen. Diese Arbeit versucht auf verschiedene Weisen zur Verbesserung der Abbildung gesellschaftlicher Prozesse in sozial-\"{o}kologischen und sozio\"{o}konomischen Modellen beizutragen.

  Diese Arbeit beginnt mit einer Modellierungsstudie des sozial-\"{o}kologischen Systems der antiken Maya auf der Halbinsel Yucatan. Diese Studie analysiert die m\"{o}glich endogene Dynamiken resultierend aus lokalem Bev\"{o}lkerungswachstum, das durch Einkommen aus Landwirtschaft und \"{O}kosystemdienstleistungen getragen wird und in dessen Folge eine \"{U}bernutzung der Umweltressourcen zu Einkommensausf\"{a}llen und damit zu Migration, R\"{u}ckgang der lokale Bev\"{o}lkerung und r\"{a}umliche Reorganisation f\"{u}hren. Die Studie zeigt auch, dass D\"{u}rreperioden von Schweregrad und Dauer, die mit pal\"{a}oklimatischen Daten \"{u}bereinstimmen, nicht in der Lage sind, dauerhafte Ver\"{a}nderungen in der Maya-Zivilisation im Modell zu bewirken. Dies steht im Einklang mit einer bestehenden Literatur, die argumentiert, dass zus\"{a}tzlich zu klimatischem Stress, interne gesellschaftliche Ver\"{a}nderungen stattgefunden haben m\"{u}ssen, um den Niedergangs und derder Maya w\"{a}hrend der Terminal-Klassik zu erkl\"{a}ren.

  Als eine m\"{o}gliche Antwort auf die Frage, wie man interne gesellschaftliche Ver\"{a}nderungen modelliert, schl\"{a}gt diese Arbeit die Differenzierung von Urteil und Handeln in der menschlichen Entscheidungsfindung vor. Insbesondere wird vorgeschlagen, den Menschen als begrenzten rationalen Entscheidungstr\"{a}ger zu modellieren, der (soziales) Lernen nutzt, um Entscheidungsheuristiken zu erwerben, die in einer bestimmten Umgebung gut funktionieren.

Anschlie{\ss}end stellt diese Arbeit ein Zwei-Sektor-Wirtschaftsmodell vor in dem der eine Sektor eine fossile Ressource f\"{u}r die wirtschaftliche Produktion nutzt, w\"{a}hrend der andere Sektor fossile freie Technologien verwendet. Die Haushalte in diesem Modell treffen Ihre Investitionsentscheidungen in der oben vorgeschlagenden Weise. Die Parameter des Modells werden anhand von historischen Daten gesch\"{a}tzt und die Modelldynamik wird in einer Reihe von numerischen Experimenten analysiert. Diese Experimente zeigen, wie in der Modell\"{o}konomie individuelle Entscheidungsfindung und soziale Dynamik die Treibhausgasemissionen nicht auf ein Niveau begrenzen k\"{o}nnen, das globale Erw\"{a}rmung \"{u}ber $1,5^{\circ}C$ verhindert. Sie zeigen aber auch, dass dies durch eine Kombination aus kollektivem Handeln und koordinierter Politik m\"{o}glich ist.

Eine Folge-Studie analysiert das soziale Lernen individueller Sparquoten in einer Ein-Sektor-Investitionswirtschaft. Hier sind die Haushalte in ein statisches soziales Netzwerk eingebettet und setzen ihre Sparquote fest, indem sie ihren Nachbarn mit dem h\"{o}chsten Konsum imitieren. Diese Studie zeigt, dass die Haushalte zu wenig sparen solange die Interaktionsrate im sozialen Lernprozess sehr hoch ist, dass sich aber die aggregierte Sparrate in der Wirtschaft der eines allwissenden, intertemporal optimierenden sozialen Planers ann\"{a}hert, wenn die Interaktionsrate sinkt. Eine sinkende soziale Interaktionsrate f\"{u}hrt au{\ss}erdem zu sprunghaft ansteigender \"{o}konomischer Ungleichheit in Form eines pl\"{o}tzlichen \"{u}bergangs von einer unimodalen zu einer stark bimodalen Verteilung des Verm\"{o}gens unter den Haushalten.

Schlie{\ss}lich schl\"{a}gt diese Arbeit eine Kombination verschiedener Moment-Closure Techniken vor, die verwendet werden k\"{o}nnen, um analytische N\"{a}herungen f\"{u}r die Dynamik vernetzter Agenten-Basierter Modelle verwendet werden k\"{o}nnen.

\end{abstract}

\cleardoublepage

\selectlanguage{english} % Switch back to English to obtain the corrected heading for table of contents
