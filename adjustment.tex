%  Import packages
\usepackage{epigraph} 		% Allows to add nice quotations at the beginning of chapters
\usepackage{amsmath}		% mathematische Erweiterung, u.a. fr den align-environment
\usepackage{amsfonts}		% fr mathematische Symbole wie \mathbb{R}
\usepackage{amssymb}
\usepackage{amsthm}
\usepackage{multirow}
\usepackage{enumerate}
%\usepackage{sidecap}
\usepackage[section]{placeins}
\usepackage{booktabs}
\usepackage{lscape}
\usepackage{longtable}
\usepackage{
  empheq, % for boxes around equations
        wrapfig, % for wrapped figures in text
        makecell, % for custom formatting in table cells
        hhline, % for custom hlines in tables
        wrapfig, % for figures wrapped in text
        rotating, % for rotated table
        tablefootnote, %for footnotes in tables
        float, % force figure placement if neccessary
        typearea,
}
\usepackage[bookmarksnumbered]{hyperref}
\usepackage[capitalize]{cleveref} % for better references
\usepackage{bibentry} % to display full bibtex entries
\nobibliography*
\usepackage{makeidx} 		% Create index
\makeindex

\usepackage{chngcntr}		% Make footnote counting continuous throughout the whole document
\counterwithout{footnote}{chapter}

%\usepackage[a-1b]{pdfx}  % Allow compliance with PDF/A standard
\usepackage[sort&compress]{natbib}

\iffalse
%
%  Define custom citation style including paper numbers
%
\usepackage[style=authoryear,maxcitenames=2,maxbibnames=100,natbib=true]{biblatex}
%\usepackage[backend=bibtex,style=authoryear,maxcitenames=2,maxbibnames=100,natbib=true, url=false]{biblatex}

\newbibmacro*{papernum}{%
  \iffieldundef{usera}{%
  }{%
    \addcomma\space
    \printfield{usera}%
  }%
}

\DeclareFieldFormat{labelyear}{%
  \stripzeros{#1}%
  \iffieldundef{extrayear}{%
    \usebibmacro{papernum}%
  }{%
  }%
}

\DeclareFieldFormat{extrayear}{%
  \iffieldnums{labelyear}
    {\mknumalph{#1}\usebibmacro{papernum}}
    {\mkbibparens{\mknumalph{#1}\usebibmacro{papernum}}}}

\addbibresource{bibliography.bib}
%%%
\fi

% easy column vectors
\newcount\colveccount
\newcommand*\colvec[1]{
        \global\colveccount#1
        \begin{pmatrix}
        \colvecnext
}
\def\colvecnext#1{
        #1
        \global\advance\colveccount-1
        \ifnum\colveccount>0
                \\
                \expandafter\colvecnext
        \else
                \end{pmatrix}
        \fi
}
 
\raggedbottom 				% Change page layout style
\allowdisplaybreaks[2]		% Allow page breaks in "align" environment

% Hurenkinder und Schusterjungen verhindern
\clubpenalty9999
\widowpenalty9999
\displaywidowpenalty9999

\usepackage{microtype}
\usepackage{hfoldsty}
\usepackage{ellipsis}
\usepackage{nicefrac}

%  Settings
\setlength{\epigraphrule}{0pt}				%  Remove epigraph rule
\setlength{\epigraphwidth}{0.50\textwidth}		%  Change epigraph width
\setlength{\mathindent}{0pt}				%  Set indent of equations
\setcounter{tocdepth}{1}					%  Set table of contents depth
\setkomafont{caption}{\sffamily\small}
\setkomafont{captionlabel}{\sffamily\bfseries\small}
\renewcommand*{\partpagestyle}{empty}

%  Definitions
\newtheorem{mydef}{Definition}	[chapter]			%  Define customized theorem environment
\newcommand{\nn}[1]{{\it #1*}}				%  Define customized style for index entries

%  Environments
\newenvironment{mytabular}{%				%  Custom tabular environment with sans-serif font
  \sffamily \small
  \tabular
}{%
  \endtabular
}

\DeclareMathOperator*{\argmax}{argmax}
\newcommand{\overbar}[1]{\mkern 1.5mu\overline{\mkern-1.5mu#1\mkern-1.5mu}\mkern 1.5mu}
